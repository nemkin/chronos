\documentclass[11pt,oneside]{article}

\usepackage[utf8]{inputenc} % Hungarian letters
\usepackage[T1]{fontenc} % No separate accents

\usepackage{graphicx} % Insert images (floats)
\usepackage{float} % Fix floats into place

\usepackage[left=1.5cm, right=1.5cm, top=1.5cm, bottom=1.5cm]{geometry} % Margins

\usepackage{booktabs} % Schedule table

\setlength{\parindent}{0pt}
\setlength{\parskip}{7pt}

\author{Viktória Nemkin\\Supervisor: Kristóf Csorba}
\title{Project Laboratory Dissertation\\LivePolyline}

\begin{document}
\maketitle
\newpage
\tableofcontents
\newpage

\section{Introduction}

CV4SensorHub is a .NET and C\# based framework that supports the development of different intelligent tools with image processing technologies for interdisciplinary applications.

GrainAutLine is one of these applications, which is a smart drawing program specializing in marble thin section and sandstone analytics designed to aid and automate the work of the user as much as possible.

LivePolyline is one of the operations this application supports, which draws around segments of the thin sections using pathfinding algorithms that stay right on the edge. The first part of this disseration will focus on implementing the class architecture for the operation using object oriented programming paradigms and design patterns to arrive at a reusable modular structure that can be quickly expanded later. In the second part we review different types of path finding algorithm, particularly A* generalizations that deal with the problem of moving targets and changing weights during search and improve the response time. The third part discusses the implementation of visual debugging techniques that help troubleshooting problems with the algorithms.


\section{Architecture design for solving large scale path finding problems}




\section{Schedule}

\begin{table}[H]
\centering
\caption{Schedule} \vspace{0.1cm}
\label{schedule}
\begin{tabular}{@{}lll@{}}
\toprule

\multicolumn{1}{c}{\textbf{Due}} &
\multicolumn{1}{c}{\textbf{Function}} &
\multicolumn{1}{c}{\textbf{Description}} \\

\midrule

\multicolumn{1}{|l|}{2017. 09. 10.} &
\multicolumn{1}{l|}{PathFinding} &
\multicolumn{1}{l|}{Breath first search} \\

\multicolumn{1}{|l|}{2017. 09. 20.} & 
\multicolumn{1}{l|}{Debugging} & 
\multicolumn{1}{l|}{Showoff} \\

&& \\

\bottomrule

\end{tabular}
\end{table}

\section{User stories}


\section{Architecture}


PathFinder
- Open list
- Closed list
- How you choose from these?

\end{document}
