\chapter*{Introduction}\addcontentsline{toc}{chapter}{Introduction}

Timetable planning for universities is a complex procedure due to the significant number of variables and conditions that have to be taken into account. In theory, there are many algorithms and methods to solve this problem but in practice, most people use last year's timetable and change manually where needed.

Similarly, at our university, this process is done entirely by hand. Software is only used to validate the result and check for collisions. Each faculty has a designated timetable creator who devises their schedule every semester. For the Faculty of Electrical Engineering and Informatics, this means manually creating 9 majors' schedules for 5000 students every single term. It is a tedious and repetitive task with no sense of reward or accomplishment and it requires a substantial amount of time to complete.

This is why I decided to automate it.

The plan is to create a software solution designed specifically for the timetabling needs of the Faculty of Electrical Engineering and Informatics of Budapest University of Technology and Economics. This software shall be capable of loading course data, the list of rooms available and teachers and then producing a timetable that meets the specific requirements of the faculty.

The first chapter will analyse the problem in detail and come up with the list of requirements that shall be met in this semester. The second chapter talks about the current market of timetabling software. The third chapter will talk about the history and background of the mathatematics required for solving the task. The fourth chapter describes the solution, the software architecture. The fifth chapter tells about the problems that arose during the semester and the solution I came up with. The sixth chapter analyses the results.
