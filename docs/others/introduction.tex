\chapter*{Introduction}\addcontentsline{toc}{chapter}{Introduction}

Timetable planning for universities is a complex procedure due to the significant number of variables and conditions that have to be taken into account. In theory, there are many algorithms and methods to solve this problem, but in practice, most people use last year's timetable and manually change where needed.

Similarly, at our university, this process is done entirely by hand. Software is only used to validate the result and check for collisions. Each faculty has a designated timetable creator who devises their schedule every semester. For the Faculty of Electrical Engineering and Informatics, this means manually creating 9 majors' schedules for roughly 5000 students every single term. It is a tedious and repetitive task with no sense of reward or accomplishment, and it requires a substantial amount of time to complete.

This is why I decided to automate it.

I aim to create a software solution designed specifically for the timetabling needs of the Faculty of Electrical Engineering and Informatics of the Budapest University of Technology and Economics. This software shall be capable of loading course data, 
a list of teachers and rooms available and then producing a timetable that meets the specific requirements of the faculty for the given semester.

The first chapter analyses the problem in detail and declares the requirements to meet during the semester. The second chapter talks about the current market of timetabling software and where these fall short on meeting these requirements. The third part addresses the mathematical background required for solving the task. The fourth chapter describes the software architecture of the solution. The fifth part discusses the problems that arose during the semester and the solutions for them. The sixth chapter examines the results.
