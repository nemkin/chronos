\chapter*{Kivonat}\addcontentsline{toc}{chapter}{Kivonat}

Az órarendtervezés elmélete rendkívül kidolgozott. Rengeteg algoritmus és matematikai elmélet létezik melyek felhasználásával órarendet lehet tervezni elméletben. Sajnos a gyakorlatban viszont az órarendeket legtöbbször kézzel készítik. Ez igaz a Budapesti Műszaki és Gazdaságtudományi Egyetemre is, ahol a minden kar saját órarendkészítő munkatársa foglalkozik ezzel a feladattal. Ez rengeteg időt és energiabefektetést igényel a részükről.

A szakdolgozatom témája az egyetemi órarendkészítés automatizálása. Mivel ez egy komplex feladat, rengeteg különféle igényt kell kielégítenie egy ideális órarendnek, ezért olyan módszert kell választani amely segítségével képesek vagyunk általánosan, matematikailag megfogalmazni ezt a problémát és olyan megoldást adni rá, mely a későbbiekben tovább bővíthető az újonnan felmerülő igényeknek megfelelően.

A választott módszer a kényszerprogramozás. A kényszerproblémák matematikai megfogalmazást adnak egy feladathoz, mely változókból, a változók értékkészletéből és egyszerű műveletekkel megadott kényszerekből, egyenletekből állnak. Ezek segítségével leírhatók tetszőlegesen bonyolult feltételrendszerek. Egy ilyen eszközkészlet segítségével programozottan lehet megoldásokat generálni a probléma leírása alapján.

\vfill

\chapter*{Abstract}\addcontentsline{toc}{chapter}{Abstract}

The theory timetable design is widely elaborated. There are plenty of algorithms and mathematical principles that can be used to create timetables in theory. Unfortunately, in practice, timetables are mostly hand-made. This is true for the Budapest University of Technology and Economics, where the faculties employ their timetable creators who deal with this task completely manually.

The subject of my dissertation is the automation of university timetable creation. This is a complex task since the ideal timetable has to meet several different requirements and needs, so we need to mathematically formulate this problem to be able to provide a solution that can be further expanded to meet emerging needs.

The method chosen is constraint programming. Constraint programming problems provide a mathematical description for a task consisting of variables, domains and mathematical formulae of constraints. They can be used to describe arbitrarily complex conditions. Using such a toolkit, one can programmatically generate solutions based on the description of the problem.

\vfill

