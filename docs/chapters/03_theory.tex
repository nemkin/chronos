\chapter{Theory}

\section{Operations research}

Operations research uses applied mathematics to give optimal solutions to complex decision-making problems maximising profitability, performance and yield or minimising loss, risk and cost of real-world objectives.

It originates from British military efforts during World War II, where it was described as ''a scientific method of providing executive departments with a quantitative basis for decisions regarding the operations under their control'', and it has grown to help a variety of industries. 

Operations research focuses on practical real-world applications, for example:

\begin{itemize}
\item Critical path analysis: identifying processes that affect the overall duration of the project.
\item Floorplanning: designing factory settings to reduce the cost of manufacturing or computer chip layouts to reduce energy cost.
\item Public transportation: determining the routes and schedules of buses so that as few buses are needed as possible.
\item Supply chain management: managing the flow of raw materials and products based on uncertain demand for the finished products.
\end{itemize}

Timetable planning is a problem that can be solved using methods from operations research.

Source: https://en.wikipedia.org/wiki/Operations\_research

\section{Constraint Satisfaction Problems}

Source: https://en.wikipedia.org/wiki/Constraint\_satisfaction\_problem

One of the techniques in operations research is Constraint Programming.

Formally, we define Constraint Satisfaction Problems as follows:

A CSP is a triplet $\{X,D,C\}$, where:

\begin{center}
\begin{tabular}{ll}
$X=\{x_1, ..., x_n\}$ & is a set of variables.\\
$D=\{d_1, ..., d_n\}$ & is a set of domains, so that $\forall{}i$, $1\leq{}i\leq{}n$, $x_i \in{} d_i$.\\
$C=\{c_1, ... c_m\}$  & is a set of constraints on the variables in $X$.\\
\end{tabular}
\end{center}

The constraints used in constraint programming are of various kinds: those used in constraint satisfaction problems (e.g. "A or B is true"), linear inequalities (e.g. "x <= 5"), and others. Constraints are usually embedded within a programming language or provided via separate software libraries.

Constraints differ from the common primitives of imperative programming languages in that they do not specify a step or a sequence of steps to execute, but rather the properties of the solution to be found. This makes constraint programming a form of declarative programming. Using this approach we can mathematically describe the problem with variables and equations. Unlike other approaches, for example using the method of colouring a bipartite graph for timetable creation this method provides flexibility with the type of features we can build into the system. With constraints we can express a wide range of requirements and we can be sure if new requests come in we will be able to include them in the system.

This is why constraint programming is a good choice for solving this problem.

