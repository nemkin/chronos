\chapter{Theory}
Operations research

Operations research uses applied mathematics to give optimal solutions to complex decision-making problems maximising profitability, performance and yield or minimising loss, risk and cost of real-world objectives.  Operations research focuses on practical applications.

It originates from British military efforts during World War II where it was described as "a scientific method of providing executive departments with a quantitative basis for decisions regarding the operations under their control" and has grown to help a variety of industries. 

Things operational research addresses:

- Critical path analysis or project planning: identifying those processes in a complex project which affect the overall duration of the project
- Floorplanning: designing the layout of equipment in a factory or components on a computer chip to reduce manufacturing time (therefore reducing cost)
- Network optimization: for instance, setup of telecommunications or power system networks to maintain quality of service during outages
- Allocation problems
- Facility location
- Assignment Problems
- Bayesian search theory: looking for a target
- Optimal search
- Routing, such as determining the routes of buses so that as few buses are needed as possible
- Supply chain management: managing the flow of raw materials and products based on uncertain demand for the finished products
- Project production activities: managing the flow of work activities in a capital project in response to system variability through operations research tools for variability reduction and buffer allocation using a combination of allocation of capacity, inventory and time
- Efficient messaging and customer response tactics
- Automation: automating or integrating robotic systems in human-driven operations processes
- Globalization: globalizing operations processes in order to take advantage of cheaper materials, labor, land or other productivity inputs
- Transportation: managing freight transportation and delivery systems (Examples: LTL shipping, intermodal freight transport, travelling salesman problem)
- Scheduling: Personnel staffing Manufacturing steps
Project tasks
Network data traffic: these are known as queueing models or queueing systems.
Sports events and their television coverage
Blending of raw materials in oil refineries
Determining optimal prices, in many retail and B2B settings, within the disciplines of pricing science
Cutting stock problem: Cutting small items out of bigger ones.

Timetable planning is one of these problems.

Source: https://en.wikipedia.org/wiki/Operations\_research

Constraint Programming/Constraint Satisfaction Problems

One of the techniques in operations research is Constraint Programming.

In computer science, constraint programming is a programming paradigm wherein relations between variables are stated in the form of constraints. Constraints differ from the common primitives of imperative programming languages in that they do not specify a step or sequence of steps to execute, but rather the properties of a solution to be found. This makes constraint programming a form of declarative programming. The constraints used in constraint programming are of various kinds: those used in constraint satisfaction problems (e.g. "A or B is true"), linear inequalities (e.g. "x <= 5"), and others. Constraints are usually embedded within a programming language or provided via separate software libraries.

Mathematically describe the problem using variables and equations (constraints) that must be true for all acceptable solutions. Then create an expression which shall be minimized to continue.

This approach was ideal because with variables and equations I can describe any constraint using mathematics which would be harder if I modelled the problem in a specific way (for example colouring of a bipartite graph). This way when new feature requests arise and new requests come in from various teachers or students I can always account for these. I just have to come up with new equations and variables and fiddle with maths to do so.

This is why I choose COP to solve this problem.

