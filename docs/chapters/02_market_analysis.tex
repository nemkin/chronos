\chapter{Market analysis}

Few timetable creator products exist on the market. The most notable ones are Timetabler (https://www.timetabler.com/) and UniTime (https://www.unitime.org/).

Timetabler is a corporate software which retails for 363.004 HUF. It is a single application that runs on Windows. This is unfortunate since we would prefer to separate the data entry and the solver from each other. Data entry will be done on a personal computer by a human, for an extended period with many interruptions. High-performance computing with expensive hardware is only required for the solver. This can run on a better quality server on campus.

Many advanced features are missing from this software that would be needed for our university. For example, you can't restrict locations for classes, can't minimise travel time between classes and there is no way to restrict the number of parallel practice sessions (where the amount of student demonstrators is limited for a department).

UniTime is an open source application available on GitHub, mainly written by Tomáš Müller from Purdue University in the United States. It is an excellent application, with many features, like a separate client for data entry and a server for running the optimiser. Sadly, the structure of higher education in the United States is very different from how it is in Hungary, and this application does not account for some basic needs our university has. Most notably they only try to minimise student class conflicts. However, we are required to eliminate conflicts in our sample curriculum for every semester. Moreover, students are considered individually and not in a year group.

Neither of these has the band system we have in place: exam timeslots, elective timeslots and timeslots for specialisations. There is no way to specify the order of classes. For example, practice sessions should follow seminars, and laboratory classes that require specialised equipment have a strict order, so the teachers don't have to shift equipment back and forth. It can't say there is no teacher assigned yet. It can't restrict the number of parallel practice sessions for departments with a small number of student demonstrators. Most notably neither of these communicates with Neptun, which is crucial for our application.

No software currently available on the market can provide every feature needed for our university and faculty. This is why we currently create the timetables manually.

I believe it makes sense to start a project to create a custom software tailored specifically for our needs and maybe in a few years it could actually be used to generate our schedules automatically.

